\documentclass{beamer}

\usepackage{listings}
\definecolor{dkgreen}{rgb}{0,0.6,0}
\definecolor{gray}{rgb}{0.5,0.5,0.5}
\definecolor{mauve}{rgb}{0.58,0,0.82}

\lstdefinestyle{myScalastyle}{
  language=scala,
  aboveskip=2mm,
  belowskip=2mm,
  showstringspaces=false,
  columns=flexible,
  basicstyle={\small\ttfamily},
  numbers=none,
  numberstyle=\small\color{gray},
  keywordstyle=\color{blue},
  commentstyle=\color{dkgreen},
  stringstyle=\color{mauve},
  frame=single,
  breaklines=true,
  breakatwhitespace=true,
  tabsize=3,
}

\lstdefinestyle{myTerminal}{
  frame=tb,
  aboveskip=2mm,
  belowskip=2mm,
  showstringspaces=false,
  columns=flexible,
  basicstyle={\small\ttfamily},
  numbers=none,
  numberstyle=\small\color{gray},
  frame=single,
  breaklines=true,
  breakatwhitespace=true,
  tabsize=3,
}

\beamertemplatenavigationsymbolsempty

\title{Create custom linters using TASTy inspection}
\author{Vincent de Haan}
\date{}

\begin{document}

\frame{\maketitle}

\frame{
  \frametitle{Who am I?}

  \begin{itemize}
    \item self-employed Scala engineer
    \item with a background in both law and mathematics
    \item currently working on smart energy meters
  \end{itemize}
}

\frame{
  \frametitle{Goals for today}

  \begin{itemize}
    \item What is a TASTy?
    \item How to inspect it?
    \item How to create a custom linter?
  \end{itemize}
}

\begin{frame}[fragile]
  \frametitle{How does this work? (Scala 2)}
  
  \begin{lstlisting}[style=myScalastyle,frame=none]
class Foo {
    def f(i: Int): Int = 1
    def g(implicit i: Int): Int = 2
}
  \end{lstlisting}
  \pause
  \begin{lstlisting}[style=myTerminal,frame=none]

> javap Foo   
Compiled from "Foo.scala"
public final class Foo {
  public static int g(int);
  public static int f(int);
}
  \end{lstlisting}
  
\end{frame}

\begin{frame}[fragile]
  \frametitle{How does this work? (Scala 2)}

  \begin{lstlisting}[style=myScalastyle,frame=none]
object FooClient {
    implicit val i: Int = 3
    
    Foo.g
    Foo.f // Does not compile
}
  \end{lstlisting}
  
  \pause
 
  How does the compiler know \texttt{g} takes an implicit?
\end{frame}

\begin

\end{document}